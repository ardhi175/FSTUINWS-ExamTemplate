\thispagestyle{empty}
\begin{center}
{\Large\bf RUBRIK PENILAIAN}
\end{center}
\begin{table}[h]
\centering
\begin{tabular}{|p{5cm}|p{10cm}|c|}
\hline
\parbox{5cm}{\bf\centering Aspek} & \parbox{10cm}{\bf\centering Kriteria} & {\bf\centering Skor} \\ \hline
 \parbox{5cm}{\vspace{0.2cm}Mekanika Kuantum 1D\vspace{0.2cm}} & \parbox{10cm}{\vspace{0.2cm} Buktikan bahwa operator $\hat{a}_-$ menurunkan swanilai energi sebesar $\hbar\omega$\vspace{0.2cm}} & 15 \\ \hline
 \parbox{5cm}{\vspace{0.2cm}Matematika Mekanika Kuantum\vspace{0.2cm}} & \parbox{10cm}{\vspace{0.2cm}Hitung peluang mendapati energi partikel sebesar $\frac{1}{2}\hbar\omega$, $\frac{3}{2}\hbar\omega$ dan $\frac{5}{2}\hbar\omega$ pada keadaan yang diberikan\vspace{0.2cm}} & 20 \\ \hline
\end{tabular}
\end{table}





