% =======================================================
% Warning
% =======================================================
%     If there is a blank page at the end of the exam,
%     remove the answer space for the last question.
% =======================================================

% Comment out this command to show only your questions,
% adding a % at the beginning of the line, e.g., %\boxedtext{\textbf{Kerjakan soal-soal di bawah ini secara mandiri. Boleh membuka buku cetak atau catatan yang dibawa sendiri. Tas, laptop dan semua perangkat komunikasi wajib diletakkan di depan.}}
\vspace{0.5cm}

\textbf{Soal 1 [Sub-CPMK-1 (5\%)]}: \lipsum[1][1-3]

\textbf{Soal 2 [Sub-CPMK-2 (5\%)]}: \lipsum[2][1-3] 

\textbf{Soal 3 [Sub-CPMK-3 (5\%)]}: \lipsum[3][1-3]

%\boxedtext{\textbf{Kerjakan soal-soal di bawah ini secara mandiri. Boleh membuka buku cetak atau catatan yang dibawa sendiri. Tas, laptop dan semua perangkat komunikasi wajib diletakkan di depan.}}
\vspace{0.5cm}

\textbf{Soal 1 [Sub-CPMK-1 (5\%)]}: \lipsum[1][1-3]

\textbf{Soal 2 [Sub-CPMK-2 (5\%)]}: \lipsum[2][1-3] 

\textbf{Soal 3 [Sub-CPMK-3 (5\%)]}: \lipsum[3][1-3]
 

% =======================================================
% PASTE YOUR QUESTIONS BELOW AND COMMENT PREVIOUS COMMAND
% =======================================================
\vspace{0.5cm}
\boxedtext{
\noindent \textbf{Kerjakan soal-soal di bawah ini secara mandiri. Boleh membuka buku cetak atau catatan yang dibawa sendiri. Tas, laptop dan semua perangkat komunikasi wajib diletakkan di depan.
}
}
\vspace{0.5cm}

\noindent\textbf{[CPMK-2 (15\%)]}\\
\noindent\textbf{Soal 1 [Sub-CPMK-4 (15\%)]}:] Jika keadaan $\psi$ bagi partikel dalam potensial Harmonik yang memenuhi persamaan $\hat{H}\psi=E\psi$ dikenai operator $\hat{a}_-$ berubah menjadi $\psi'=\hat{a}_-\psi$, buktikan bahwa berlaku
\begin{equation}
    \hat{H}\psi'=E'\psi',
\end{equation}
dengan $E'=E-\hbar\omega$ dan
\begin{equation}
    \hat{a}_{\pm}=\frac{1}{\sqrt{2m}}\biggl(\frac{\hbar}{i}\frac{d}{dx}\pm im\omega x\biggr),\qquad \hat{H}=\hat{a}_-\hat{a}_+-\frac{1}{2}\omega\hbar.
\end{equation}\\


\noindent\textbf{[CPMK-3 (20\%)]}\\
\noindent\textbf{Soal 2 [Sub-CPMK-5+6 (20\%)]}:] Andaikan sebuah partikel dalam pengaruh potensial Harmonik  pada suatu waktu dipersiapkan untuk berada pada keadaan normal 
\begin{equation}
    \psi(x)=\frac{1}{\sqrt{3}}\psi_1(x)+\alpha\psi_2(x),\qquad \alpha : \textrm{konstanta},
\end{equation} 
dengan
\begin{equation}
    \psi_n(x)=\biggl(\frac{m\omega}{\pi\hbar}\biggr)^{1/4}\frac{1}{\sqrt{2^n n!}}H_n(\xi)e^{-\xi^2/2},
\end{equation}
dengan $H_n(\xi)$ adalah polinom Hermite orde $n$ dan 
\begin{equation}
    \xi=\sqrt{\frac{m\omega}{\hbar}}x.
\end{equation}
Pada pengukuran energi di waktu berikutnya, tentukan peluang mendapati partikel memiliki energi sebesar :\\[0.2cm]
\noindent\textbf{a}. $\frac{1}{2}\hbar\omega$\\[0.2cm]
\noindent\textbf{b}. $\frac{3}{2}\hbar\omega$\\[0.2cm]
\noindent\textbf{c}. $\frac{5}{2}\hbar\omega$\\
