\begin{longtable}{|lp{14.7cm}|}
\hline
\rowcolor{lightgray}\multicolumn{2}{|l|}{\bf CAPAIAN PEMBELAJARAN PRODI YANG DIDUKUNG}    \\ \hline
\multicolumn{1}{|l|}{CPL-03} & \parbox{14.7cm}{\vspace{0.2cm} Mampu mendeduksi konsep-konsep fisis berdasarkan prinsip-prinsip pokok fisika dengan menggunakan perangkat matematik serta penerapannya dalam pengembangan IPTEK\vspace{0.2cm}} \\ \hline
\multicolumn{1}{|l|}{CPL-10} & \parbox{14.7cm}{\vspace{0.2cm} Mampu menganalisis berbagai solusi alternatif yang ada terhadap permasalahan fisis dan menyimpulkannya untuk pengambilan keputusan yang tepat\vspace{0.2cm}} \\ \hline
\multicolumn{1}{|l|}{CPL-09} & \parbox{14.7cm}{\vspace{0.2cm} Mampu menghasilkan model matematis atau model fisis yang sesuai dengan hipotesis atau prakiraan dampak dari fenomena yang menjadi subyek pembahasan\vspace{0.2cm}} \\ \hline
\multicolumn{1}{|l|}{CPL-11} & \parbox{14.7cm}{\vspace{0.2cm} Mampu memprediksi potensi penerapan perilaku fisis dalam teknologi\vspace{0.2cm}} \\ \hline
\multicolumn{1}{|l|}{CPL-04} & \parbox{14.7cm}{\vspace{0.2cm} Mampu menguasai dan mengimplementasikan integrasi ilmu keislaman dengan ilmu fisika\vspace{0.2cm}}
 \\ \hline
\rowcolor{lightgray}\multicolumn{2}{|l|}{\bf CAPAIAN PEMBELAJARAN MATA KULIAH}    \\ \hline
\multicolumn{1}{|l|}{CPMK-01} & \parbox{14.7cm}{\vspace{0.2cm} Mampu menjelaskan latar belakang teori mekanika kuantum \vspace{0.2cm}} \\ \hline
\multicolumn{1}{|l|}{CPMK-02} & \parbox{14.7cm}{\vspace{0.2cm} Mampu menjelaskan penerapan konsep mekanika kuantum dalam berbagai kasus 1 dimensi \vspace{0.2cm}} \\ \hline
\multicolumn{1}{|l|}{CPMK-03} & \parbox{14.7cm}{\vspace{0.2cm} Mampu menjelaskan dasar matematika untuk mekanika kuantum serta menerapkannya dalam penyusunan postulat
mekanika kuantum \vspace{0.2cm}} \\ \hline
\multicolumn{1}{|l|}{CPMK-04} & \parbox{14.7cm}{\vspace{0.2cm} Mampu menjelaskan mekanika kuantum 3D \vspace{0.2cm}} \\ \hline
\multicolumn{1}{|l|}{CPMK-05} & \parbox{14.7cm}{\vspace{0.2cm} Mampu menjelaskan konsep time independent perturbation theory \vspace{0.2cm}} \\ \hline
\multicolumn{1}{|l|}{CPMK-06} & \parbox{14.7cm}{\vspace{0.2cm} Mampu menjelaskan fine structure dan zeeman effect \vspace{0.2cm}} \\ \hline
\multicolumn{1}{|l|}{CPMK-07} & \parbox{14.7cm}{\vspace{0.2cm} Mampu menjelaskan konsep time dependent perturbation theory dalam mekanika kuantum \vspace{0.2cm}} \\ \hline
\end{longtable}


